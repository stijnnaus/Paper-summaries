\begin{document}

\toc

\section{First 5}

\subsection{\textit{“Continuing emissions of methyl chloroform from Europe”}, Krol et al (2003)}
Using TM5, they show that the evolution (unexpectedly slow decline) of MCF is explained better by additional emissions over certain parts of Europe (Total $>$ 20 Gg/yr, compared to assumed $<$ 1 Gg/yr), than by a negative OH trend. This is mainly because they better explain the relatively large spatial (vertical) and temporal variability. These emissions are mostly important to determine the OH distribution and subsequently strongly influence the NH/SH ratio. Lot of info on MCF.
Why would MCF emissions be so much restricted to certain periods in the year? 

\subsection{\textit{“What can $^{14}$CO measurements tell us about OH?”}, Krol et al (2008)}
They first use TM5 to simulate the $^{14}$CO distribution to closely resemble measurements. The largest sensitivity is to source function and next model resolution.
Next they look at the sensitivity of the simulated $^{14}$CO distribution to OH. Sensitivity is highest to high-latitude OH, in contrast to MCF. Due to a shorter lifetime compared to MCF, $^{14}$CO gives information about the OH field on smaller spatial scales. Because of the distribution of measurement stations, this means $^{14}$CO can mainly constrain OH at high latitudes. Contains a clear explanation of the processes affecting $^{14}$CO on synoptic scales. The main issue in using $^{14}$CO as tracer are the necessity for correctly simulating transport on small spatial and temporal scales, and data availability.

Adjoint approach is explained: Release a pulse at a measurement station and integrate it back through time. Then you get the sensitivity of the measurements to pulses released at some earlier time at all grid cells in 1 simulation.

\subsection{\textit{"Small interannual variability of global atmospheric hydroxyl radical”}, Montzka et al (2011)}
This study estimates a small interannual variability in OH (derived from MCF) compared to previous studies. This is probably because it looks at a period ($>$ 1997) where MCF emissions steeply declined, so that emissions are well known. Subsequently, the high previous estimates are likely because emissions of MCF are less well known than assumed: OH interannual variability is highly sensitive to MCF emissions and their timing.
5 more tracers were also used to validate estimates from MCF. Though their sources are much less well known, the results are similar to those from MCF, especially after 1997.

\subsection{\textit{“A multi-year methane inversion using SCIAMACHY, accounting for systematic errors using TCCON measurements”}, Houweling et al. (2014)}
The study addresses the poorly constrained methane budget and especially its distribution in the tropics. This is done by inverse modelling, using satellite data. Satellite data has good coverage compared to ground-based data, but is known to contain large systematic errors.
Previous studies introduced bias parameters to reduce the difference between model results and observations, possibly transmitting systematic errors in the satellite data into models. For this reason, ground-based measurements are used to identify the systematic errors. 

\subsection{“Observational evidence for interhemispheric hydroxyl-radical parity”, Patra et al. (2014)}
The meridional OH gradient is important for estimating hemispheric source and sink magnitudes of gases and aerosols that interact with OH (e.g. NOx, methane). Most models predict an OH ratio NH/SH significantly higher than one, mostly biased by high/low ozone in NH/SH (OH source). Models based on MCF show a ratio close to one and some models show ratios even lower.

In this study they simulate MCF and SF6 using 2 models with different NH/SH ratios (0.87 and 1.18). The temporal evolution of MCF is simulated well and not significantly differently by both, suggesting that global OH concentrations are well known. The MCF gradient is better captured by the lower ratio. Source distribution and model distribution do not significantly affect the MCF gradient. The gradient can be simulated by the high ratio, if increased chemical loss and increased emissions are considered, but this reduces the match with the seasonal cycle. This shows that emissions in 2004-2011 are quite well known. Sensitivity experiments to the OH NH/SH gradient are carried out and indicate a predicted ratio of 0.97 $\pm$ 0.12.

Overall: Further investigation is required. Also, using the higher ratio requires shifting CH4 emissions, suggesting that CH4 emissions, even if uncertain, can help in validating or constraining OH estimates.



















































\end{document}